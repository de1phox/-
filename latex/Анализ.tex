\section{Анализ предметной области}

\subsection{Характеристики растений}
Декоративные растения — это разнообразная группа растений, включающая как культивируемые, так и дикие виды деревьев, кустарников, многолетних и однолетних растений, злаков и луковичных растений, которые принадлежат к различным ботаническим семействам. Их уникальные декоративные качества делают их незаменимыми для озеленения садов, парков и скверов, а также для украшения зданий и внутренних помещений.

Внешний облик декоративных растений играет ключевую роль в определении их декоративных характеристик. Важными аспектами являются живописность кроны, изящество силуэта, длительность и обильность цветения, изменчивая окраска в течение года, способность выдерживать неблагоприятные условия и климатические изменения.

Декоративные злаки — это растения семейства злаковых, которые используются для создания групповых посадок и бордюров, газонов, а также для составления букетов. Они обладают привлекательной текстурой и формой, что делает их популярными элементами ландшафтного дизайна \cite{kingsberry}.

Луковичные растения представляют собой обширную группу декоративных растений, различающихся по цвету, размеру, форме цветков и времени цветения. В открытом грунте они цветут с весны до осени, а благодаря зимней выгонке способны цвести практически круглый год. Эти растения могут использоваться для выращивания в помещении и для создания букетов.

С точки зрения распространения по миру, луковичные растения представляют одну из самых многочисленных групп. Они включают в себя преимущественно лилейные и амариллисовые семейства \cite{belyaevskaya}.

Луковичные растения могут размножаться как вегетативным способом, так и семенами. Лилии и гиппеаструмы обычно размножаются семенами, этот метод также используется при создании новых сортов.

Растения, образующие луковицы, требуют особой почвы. Им подходят легкие, проницаемые и влагоемкие грунты с нейтральной или слабощелочной реакцией (pH не ниже 6–7). Последнее особенно важно для тюльпанов. Для лилий подходят легкие глинистые почвы с хорошей дренажной системой, практически без извести. Гиацинты и тюльпаны предпочитают песчаные почвы, богатые питательными веществами, с нейтральной кислотностью. Тяжелые глинистые почвы с плохой водопроницаемостью, а также очень бедные песчаные почвы не подходят для луковичных растений \cite{doroshenko}.

Декоративные деревья и кустарники бывают хвойные и лиственные, вечнозелёные или с опадающей листвой. Их красота зависит от внешних характеристик, таких как форма кроны, цвет и структура листьев, а также цвет и размер цветов и плодов. Подходящие условия для роста помогают сохранить красоту растений, которая меняется вместе с возрастом и сменой времен года в зависимости от генетических особенностей вида.

У древесных растений существуют два способа размножения: семенной и вегетативный. Семенное размножение осуществляется через семена, а вегетативное — через черенки, отводки, корневые отпрыски, прививки или деление куста \cite{berd}.

При создании групп и игры с контрастными цветами важную роль играет окраска листьев деревьев. Она может усилить общее впечатление от кроны, добавить плотности и объема. В течение сезона меняется интенсивность окраски листьев. Форма и размер листьев также влияют на декоративность растений, особенно если они находятся рядом с дорожками. Растения с красивыми и необычными листьями имеют большую ценность в ландшафтном дизайне, если они не характерны для местной флоры.

В период цветения декоративные растения приобретают особую привлекательность. Они отличаются разнообразием форм, цветов, размеров соцветий, ароматами, продолжительностью и временем цветения. Правильный выбор цветущих растений поможет создать композицию с длительным цветением.

Жизненный цикл декоративных растений — это онтогенез, или индивидуальное развитие растения от его появления из оплодотворенной яйцеклетки или вегетативной почки (черенка с почками, корневого отпрыска и т.п.) до отмирания.

\subsubsection{Многолетние растения}

Характерной чертой многолетних травянистых растений является способность переживать зиму на открытой земле, ежегодно проходить через все стадии развития (рост, цветение, плодоношение) и продолжать этот цикл на протяжении многих лет благодаря специальным органам, предназначенным для зимовки (корень, корневище, луковица, ползучие стебли и другие).

У большинства декоративных многолетников надземная часть каждый год отмирает, но на следующий сезон восстанавливается за счет измененного подземного стебля в виде корневищ, луковиц, клубнелуковиц (например, луковичные растения, пионы, флоксы, рудбекия, акониты и другие) \cite{viyginaOpen}.

У других растений надземные побеги не отмирают, а сохраняются зимой, когда все жизненные процессы, как у древесных растений, замедляются. С приходом весны и благоприятных условий из почек начинают развиваться новые вегетативные и генеративные побеги, которые цветут и образуют семена.

У некоторых видов растений стебель и корень, помимо своих основных функций, также выполняют задачи вегетативного размножения и защиты растений зимой.

Существуют три основных типа подземных стеблей: луковица, клубень (клубнелуковица) и корневище.

Луковица – это видоизменённый побег с плотным коротким стеблем и мясистыми листьями, имеющими вид чешуек и запасающими воду и питательные вещества \cite{belyaevskaya}.

Клубень -- это укороченный, значительно утолщенный подземный побег, который запасает питательные вещества. Клубни, имеющие форму луковицы, называются клубнелуковицами.

Корневище -- это долговечное стеблевое образование, внешне напоминающее корень, но отличающееся наличием дополнительных корней и почек. На его поверхности можно видеть следы прикрепления листьев в виде коричневых чешуек и остатки умерших побегов. Корневище может быть горизонтальным, утолщенным (как у ириса, бадана, ландыша), растущим вертикально или наклонно вниз (как у астильбы, примулы, функии, диклитры) \cite{aldohina}.

У многолетних стержнекорневых растений основной корень не замещается дополнительными, а сохраняется на протяжении всей жизни растения. На его поверхности могут быть небольшие корешки с дополнительными почками (корневые отпрыски), используемые для размножения.

Многолетние растения, благодаря своим биологическим особенностям, способны выдерживать низкие температуры зимой. Молодые экземпляры имеют невысокую степень морозоустойчивости, но по мере их развития эта способность увеличивается. Однако по мере завершения жизненного цикла у растений морозоустойчивость снова снижается. Для сохранения и развития этого ценного биологического свойства растений важна своевременная подготовка к зиме, проведение осенних и ранневесенних посевов в открытый грунт. Не столько низкие температуры, сколько резкие перепады могут нанести вред многолетникам. Особенно опасны для них зимы без снега, с оттепелями и долгими затоплениями поверхности почвы водой. Поэтому для выращивания многолетников, особенно тех, что длительное время находятся на одном месте (например, пионы, флоксы, функии и другие), необходимо выбирать ровные, хорошо дренированные участки без застоя воды.

При создании декоративных миксбордеров (цветников-бордюров из смешанных растений разных размеров и оттенков) важно учитывать морозоустойчивость растений и подбирать ассортимент с учетом конкретных почвенно-климатических условий каждой зоны. Только таким образом можно обеспечить успешное процветание и красоту растений в течение долгого времени \cite{karpisonova}.

\subsubsection{Двулетние растения}

В цветоводстве под двулетниками понимаются растения, которые по своей природе являются многолетниками, но достигают пика декоративного эффекта в период своего второго года жизни. К их числу относятся такие виды, как виола, колокольчики, наперстянка, маргаритки и незабудки.

Двулетники обладают разнообразными биологическими и декоративными характеристиками, благодаря чему их применяют для украшения садов и цветников в начале весны, в качестве материала для создания цветочных композиций, для выращивания в горшках, а также для украшения помещений и пристенного озеленения \cite{aldohinaFlower}.

В отличие от однолетников, двулетники не завершают свой жизненный цикл за один сезон. В первый год они формируют мощную розетку листьев, в то время как во втором году достигают максимального цветущего состояния и дают плоды. В ходе третьего зимнего сезона большая часть растений гибнет, но из семян, оставшихся после увядания, вновь появляются новые растения, которые также достигают пика цветения во втором году своей жизни. Таким образом, участки, занятые двулетниками, способны к самовозобновлению и могут сохранять свою декоративность на протяжении многих лет, со временем изменяя свой облик.

При сильной засухе двулетники могут зацвести и дать небольшое количество семян в первый год жизни.

\subsubsection{Однолетние растения}

К группе однолетников относятся однолетние цветочные культуры, которые проходят все стадии развития за один сезон. Это астры, ноготки, бархатцы, васильки, алиссумы, маки, флоксы летние и другие растения.

В год посева они цветут, дают семена и погибают. Продолжительность их жизни — всего несколько месяцев в году. Однако по разнообразию форм, яркости окраски цветков и длительности цветения многие летники превосходят другие цветочные растения. Они обладают исключительным многообразием по всем декоративным особенностям и предоставляют продолжительное время  для срезки в букеты — от апреля до самых морозов. Кроме того, летники имеют ещё одно положительное свойство — их можно пересаживать в цветущем состоянии.

К однолетникам причисляют некоторые грунтовые многолетники, которые зацветают в первый же год, но не переносят зиму в наших условиях. К ним относятся львиный зев, петуния, вербена и другие растения, которые на своей родине живут по несколько лет.

По декоративным качествам и применению в озеленении летники можно разделить на цветущие и лиственно-декоративные (орнаментальные), бордюрные и вьющиеся.

Большой видовой и сортовой ассортимент однолетних растений позволяет выбрать множество однолетников, которые можно размножать безрассадным способом — посевом семян в открытый грунт. Этот способ отличается доступностью и дешевизной.

Посев семян летников можно производить осенью — после окончания периода длительных потеплений (вторая половина ноября — начало декабря), весной — до наступления вегетационного периода (вторая половина апреля) или летом. Весенний посев в большинстве случаев даёт лучшие результаты, чем осенний, поэтому надежнее сеять летники весной.

Непосредственно в открытый грунт весной и осенью высевают семена многих однолетних растений, которые достаточно холодостойки. Более теплолюбивые виды следует высевать только после того, как прекратятся сильные весенние заморозки по утрам.

Также в открытый грунт высевают семена тех растений, которые не переносят пересадку (пикировку) из-за того, что их корневая система не может быстро восстановить всасывающие волоски, которые развиваются только на глубоко уходящих в почву корневых ответвлениях (мак, левкой, резеда, люпин, настурция, эшшольция).

В открытый грунт также высевают семена быстрорастущих видов растений, таких как декоративная фасоль, ноготки, акроклинум, кореопсис и бархатцы.

Для посева однолетних декоративных растений лучше всего использовать крупные отборные семена. Отбор крупных семян имеет огромное преимущество — они содержат большой запас питательных веществ, сеянцы быстро развиваются и образуют сильнорослые устойчивые растения с крупными соцветиями и цветками, которые лучше сопротивляются грибковым заболеваниям и вредителям. Кроме того, в таких случаях наблюдается более раннее цветение \cite{aleksandrova}.

\subsection{Особенности продажи растений}
Продавцы несут ответственность перед покупателями и должны соблюдать определённые правила. Они обязаны предоставить информацию о своём фирменном наименовании, месте нахождения организации и режиме работы. Эта информация должна быть размещена на вывеске магазина.

Если продавец является индивидуальным предпринимателем, он также должен сообщить покупателю о своей государственной регистрации и наименовании органа, который её провёл.

У продавцов должна быть книга отзывов и предложений, которую они обязаны предоставить покупателю по первому требованию. Это помогает улучшить качество обслуживания и создать положительную обратную связь.

Правила продажи различных видов товаров должны быть наглядно и доступно представлены. Это важно для обеспечения прозрачности и доверия со стороны покупателей.

Продавцы обязаны своевременно предоставлять покупателям достоверную информацию о товарах. Эта информация должна содержать наименование товара, сведения о стране происхождения (для импортных товаров), подтверждении соответствия, основных потребительских свойствах товара. Также необходимо указать цену и условия приобретения товара, информацию о разрешении на ввоз определённых видов дикорастущих растений в Российскую Федерацию \cite{rf}.

Если кассовый чек не содержит видового названия и количества растений, то покупателю должен быть передан товарный чек с этой информацией.

Продавцы обязаны предупредить покупателей о недостатках товаров не только устно, но и письменно.
Растения являются непродовольственными товарами, которые не подлежат возврату или обмену на аналогичный товар другого размера, формы, цвета и т. д., если они не имеют недостатков. Перед покупкой растения необходимо тщательно проверить его и убедиться в соответствии вашим требованиям. 
Некоторые растения могут быть опасными и ядовитыми, особенно для детей и домашних животных. При продаже таких растений продавцы также должны предоставить информацию о возможной опасности, чтобы покупатели были внимательны и предприняли необходимые меры предосторожности.


\subsection{Особенности заказа растений через Интернет}

Продажа семян и саженцев по почте — это удобный и популярный способ покупки в России. Больше не нужно обходить множество магазинов в поисках нужных сортов и гибридов: покупатели могут спокойно и внимательно изучить каталог или сайт интернет-магазина, не выходя из дома.

Во многих регионах страны выбор семян и посадочного материала ограничен, а качество и ассортимент уступают тем, что представлены в крупных интернет-магазинах и каталогах компаний. Поэтому жители регионов часто заказывают товары по почте, так как не могут найти нужный сорт или вид растения в обычных магазинах.

Дистанционная торговля семенами и посадочным материалом может осуществляться двумя способами: через каталоги и через интернет-магазины. До недавнего времени основным видом торговли была продажа через каталоги. С развитием интернета в нашей стране, на смену этому виду торговли пришла торговля через интернет-магазины.

Самый популярный и простой товар с точки зрения хранения и доставки — это семена \cite{krivko}. Их можно заказывать круглый год, многие огородники планируют свои посевы ещё с осени и делают заказы заранее, в октябре-декабре. Такие посылки приходят адресату до Нового года. Однако пик заказов обычно приходится на январь-февраль. Заказывая семена, всегда нужно учитывать сроки посева каждой культуры. Затем отнимается 3-5 дней на комплектацию заказа и 5-20 дней на доставку почтой, в зависимости от региона. В итоге получается, что заказ семян следует делать минимум за месяц до посева \cite{kaygorodtseva}.

Второй по популярности товар — это луковичные и многолетние растения. Этот вид товара имеет два сезона продаж — весна и осень. Весной предлагается огромный ассортимент многолетних растений, таких как астильбы, хосты, пионы травянистые, гейхеры, а также луковичных растений, таких как гладиолусы, лилии, амариллисы. Срок приёмки заказов с ноября по апрель, отправка заказов начинается в марте. Осенний сезон более короткий и предлагает тюльпаны, нарциссы, крокусы, гиацинты и различные мелколуковичные растения. Заказы принимаются с июля по август, отправка осуществляется с августа по сентябрь.

Саженцы плодовых и декоративных кустарников и деревьев — следующий по популярности товар. Приём заказов начинается с ноября, а рассылка стартует после 8 марта.

Саженцы кустарников и деревьев, а также луковичные и многолетние растения — это более специфический продукт, чем семена. Для сохранения качества им нужны определённые температурные условия.
При выборе оплаты наложенным платежом покупатель оплачивает посылку при получении на почте. Стоимость посылки состоит из нескольких компонентов: стоимости товара, стоимости доставки, которая зависит от веса посылки, расстояния и выбранного способа доставки, а также комиссии за обработку наложенного платежа, которую взимает почтовая служба или курьерская компания.

При выборе предоплаты покупатель заполняет квитанцию в каталоге или получает её на электронную почту. Отправка заказа осуществляется после поступления денег на расчётный счёт продавца. При получении товара на почте покупатель оплачивает только расходы за почтовую пересылку.

