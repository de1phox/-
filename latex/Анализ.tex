\section{Анализ предметной области}
\subsection{История интернет-магазинов}
Прообраз современных интернет-магазинов появился задолго до создания мировой паутины. В 1979 году Михаэль Олдрич, занимавшийся поставками коммуникационных сетей в Великобританию, изобрёл телемагазин «videotex». Эта технология позволяла транслировать каталоги с товарами и контактными данными для связи с продавцом.
Первый интернет-магазин, похожий на современные, появился в 1992 году. Чарльз Стэк создал интернет-магазин книг. В то время книги покупали по бумажным каталогам с ценами и описаниями. Чарльз оцифровал эти каталоги и перенёс их в формат, понятный для интернета. Книги не имели срока годности и не меняли спрос из-за модных тенденций. Многие интернет-магазины, появившиеся в начале 90-х, начинали свою деятельность с продажи печатной продукции.
В 1994 году Джефф Безос придумал концепцию проекта «Amazon». Изначально в нём можно было купить только книги, но со временем ассортимент сильно расширился. Через интернет стали продавать больше вещей, которые раньше можно было купить только в обычном магазине. Позже «Amazon» стал предлагать постельное бельё, одежду и бытовую технику. Хоть код интернет-ресурса имел изъяны (например, можно было заказать отрицательное количество товара), этот магазин получил всеобщее признание.
Первый российский интернет-магазин появился в 1997 году в Москве, он был полностью книжным, как и его предшественники. Через год в стране появились первые интернет-проекты по интернет-банкингу и автоматические интернет-шлюзы для обработки заказов за секунды. Российский интернет-трейдинг сделал резкий рывок вперёд, который определил его дальнейшее развитие.
Появление интернет-магазинов стало настоящей революцией в сфере розничной торговли. Они дали покупателям возможность выбирать из огромного ассортимента товаров, сравнивать цены и делать покупки в любое удобное время.
Кроме того, благодаря интернет-магазинам удалось сократить расходы на аренду торговых площадей и оплату труда персонала. Это позволило снизить цены на товары, делая их более доступными для широкого круга потребителей.


\subsection{Особенности продажи растений}
Продавцы несут ответственность перед покупателями и должны соблюдать определённые правила. Они обязаны предоставить информацию о своём фирменном наименовании, месте нахождения организации и режиме работы. Эта информация должна быть размещена на вывеске магазина.
Если продавец является индивидуальным предпринимателем, он также должен сообщить покупателю о своей государственной регистрации и наименовании органа, который её провёл.
У продавцов должна быть книга отзывов и предложений, которую они обязаны предоставить покупателю по первому требованию. Это помогает улучшить качество обслуживания и создать положительную обратную связь.
Правила продажи различных видов товаров должны быть наглядно и доступно представлены. Это важно для обеспечения прозрачности и доверия со стороны покупателей.
Продавцы обязаны своевременно предоставлять покупателям достоверную информацию о товарах. Эта информация должна содержать наименование товара, сведения о стране происхождения (для импортных товаров), подтверждении соответствия, основных потребительских свойствах товара. Также необходимо указать цену и условия приобретения товара, информацию о разрешении на ввоз определённых видов дикорастущих растений в Российскую Федерацию.
Если кассовый чек не содержит видового названия и количества растений, то покупателю должен быть передан товарный чек с этой информацией.
Продавцы обязаны предупредить покупателей о недостатках товаров не только устно, но и письменно.
Растения являются непродовольственными товарами, которые не подлежат возврату или обмену на аналогичный товар другого размера, формы, цвета и т. д., если они не имеют недостатков. Перед покупкой растения необходимо тщательно проверить его и убедиться в соответствии вашим требованиям. 
Некоторые растения могут быть опасными и ядовитыми, особенно для детей и домашних животных. При продаже таких растений продавцы также должны предоставить информацию о возможной опасности, чтобы покупатели были внимательны и предприняли необходимые меры предосторожности.


\subsection{Особенности заказа растений через Интернет}

Продажа семян и саженцев по почте — это удобный и популярный способ покупки в России. Вам больше не нужно обходить множество магазинов в поисках нужных сортов и гибридов. Вы можете спокойно и внимательно изучить каталог или сайт интернет-магазина, не выходя из дома.
Во многих регионах страны выбор семян и посадочного материала ограничен, а качество и ассортимент уступают тем, что представлены в крупных интернет-магазинах и каталогах компаний. Поэтому жители регионов часто заказывают товары по почте, так как не могут найти нужный сорт или вид растения в обычных магазинах.
Дистанционная торговля семенами и посадочным материалом может осуществляться двумя способами: через каталоги и через интернет-магазины. До недавнего времени основным видом торговли была продажа через каталоги. С развитием интернета в нашей стране, на смену этому виду торговли пришла торговля через интернет-магазины.
Самый популярный и простой товар с точки зрения хранения и доставки — это семена \cite{krivko}. Их можно заказывать круглый год, многие огородники планируют свои посевы ещё с осени и делают заказы заранее, в октябре-декабре. Такие посылки приходят адресату до Нового года. Однако пик заказов обычно приходится на январь-февраль. Заказывая семена, всегда нужно учитывать сроки посева каждой культуры. Затем отнимайте 3-5 дней на комплектацию заказа и 5-20 дней на доставку почтой, в зависимости от региона. В итоге получается, что заказ семян следует делать минимум за месяц до посева.
Второй по популярности товар — это луковичные и многолетние растения. Этот вид товара имеет два сезона продаж — весна и осень. Весной предлагается огромный ассортимент многолетних растений, таких как астильбы, хосты, пионы травянистые, гейхеры, а также луковичных растений, таких как гладиолусы, лилии, амариллисы. Срок приёмки заказов с ноября по апрель, отправка заказов начинается в марте. Осенний сезон более короткий и предлагает тюльпаны, нарциссы, крокусы, гиацинты и различные мелколуковичные растения. Заказы принимаются с июля по август, отправка осуществляется с августа по сентябрь.
Саженцы плодовых и декоративных кустарников и деревьев — следующий по популярности товар. Приём заказов начинается с ноября, а рассылка стартует после 8 марта.
Саженцы кустарников и деревьев, а также луковичные и многолетние растения — это более специфический продукт, чем семена. Для сохранения качества им нужны определённые температурные условия.
При выборе оплаты наложенным платежом покупатель оплачивает посылку при получении на почте. Стоимость посылки состоит из нескольких компонентов: стоимости товара, стоимости доставки, которая зависит от веса посылки, расстояния и выбранного способа доставки, а также комиссии за обработку наложенного платежа, которую взимает почтовая служба или курьерская компания.
При выборе предоплаты покупатель заполняет квитанцию в каталоге или получает её на электронную почту. Отправка заказа осуществляется после поступления денег на расчётный счёт продавца. При получении товара на почте покупатель оплачивает только расходы за почтовую пересылку.

