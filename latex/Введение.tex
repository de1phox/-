\section*{ВВЕДЕНИЕ}
\addcontentsline{toc}{section}{ВВЕДЕНИЕ}

Питомник растений – это специально созданный участок, где выращивают и продают различные виды и сорта растений. Они могут быть как крупными промышленными предприятиями, так и небольшими семейными бизнесами. Главная задача питомников — производить качественные растения с хорошо развитой корневой системой, которые могут быстро адаптироваться к новым условиям роста. Растения в питомниках продаются как оптом, так и в розницу.

Из-за больших размеров и требований к окружающей среде, питомники не всегда удобно располагать в местах с большим скоплением людей. Однако они должны быть расположены вблизи населённых пунктов, чтобы обеспечить удобный доступ для покупателей и крупной техники.

Интернет-магазин позволяет привлекать покупателей из разных городов и требует гораздо меньше средств, чем открытие филиалов и привлечение дистрибьюторов. Кроме того, покупка через Интернет занимает у покупателя меньше времени, что увеличивает вероятность её совершения. Также есть возможность принимать заказы по ночам, когда физические магазины обычно не работают.

Ещё одним преимуществом создания интернет-магазина является возможность использовать маркетинговые инструменты для продвижения. Благодаря сайту можно реализовать таргетированную рекламу, поисковую оптимизацию (SEO), аналитику и многое другое.

Сайт является лицом компании и может существенно повысить её имидж. Любой пользователь Интернета сможет получить необходимую информацию о компании в любое время. На сайте можно найти контактные телефоны, адрес и электронную почту для связи с компанией. Сейчас большинство клиентов узнают о её существовании именно через сайт. Поэтому сайт можно назвать лучшей рекламой.

\emph{Цель настоящей работы} – разработка web-сайта для продажи растений через Интернет. Для достижения поставленной цели необходимо решить \emph{следующие задачи:}
\begin{itemize}
	\item провести анализ предметной области;
	\item разработать концептуальную модель web-сайта;
	\item спроектировать web-сайт;
	\item реализовать сайт средствами web-технологий.
\end{itemize}

\emph{Структура и объем работы.} Отчет состоит из введения, 4 разделов основной части, заключения, списка использованных источников, 2 приложений. Текст выпускной квалификационной работы равен \formbytotal{page}{страниц}{е}{ам}{ам}.

\emph{Во введении} сформулирована цель работы, поставлены задачи разработки, описана структура работы, приведено краткое содержание каждого из разделов.

\emph{В первом разделе} на стадии описания технической характеристики предметной области приводится сбор информации о характеристиках растений и специфике продажи товаров через Интернет.

\emph{Во втором разделе} на стадии технического задания приводятся требования к разрабатываемому сайту.

\emph{В третьем разделе} на стадии технического проектирования представлены проектные решения для web-сайта.

\emph{В четвертом разделе} приводится спецификация контроллеров приложения, производится тестирование разработанного сайта.

В заключении излагаются основные результаты работы, полученные в ходе разработки.

В приложении А представлен графический материал.
В приложении Б представлены фрагменты исходного кода.

